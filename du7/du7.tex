\documentclass[12pt, a4paper]{article}
\usepackage[margin=1in]{geometry}
\usepackage[utf8x]{inputenc}
\usepackage{indentfirst} %indentace prvního odstavce
\usepackage{mathtools}
\usepackage{amsfonts}
\usepackage{amsmath}
\usepackage{amssymb}
\usepackage{graphicx}
\usepackage{enumitem}
\usepackage{subfig}
\usepackage{float}
\usepackage[czech]{babel}
\usepackage{mathdots}

\begin{document}
\begin{center}
\large NMMB430 - DÚ 7

\normalsize Jan Oupický
\end{center}
\vspace{1\baselineskip}

\section{}
Denote $E_1: y^2=x^3+3x+1, E_2: y^2=x^3+3x-1$. Since $-1 \notin (\mathbb{F}_{31}^*)^2$ and $j(E_1)=j(E_2)$ then $E_2$ is a quadratic twist of $E_1$. Therefore we have the equality
\begin{gather*}
|E_1(\mathbb{F}_{31})|+|E_2(\mathbb{F}_{31})|=64
\end{gather*}

Also using Hasse theorem we know that $21 \leq |E_1(\mathbb{F}_{31})|,|E_2(\mathbb{F}_{31})| \leq 43$.

Let's calculate the division polynomials for $E_1$ and factor them in $\mathbb{F}_{31}[x]$:
\begin{gather*}
\phi_1= 1\\
\phi_2 = 2y\\
\phi_3 = 3x^4+18x^2+12+22 = 3(x+1)(x^3+30x^2+7x+28)\\
\phi_4 = 4y(x^6+15x^4+20x^3+17x^2+19x+27)\\
\phi_5 = 5x^{12}+8x^9+16x^8+7x^7+30x^6+29x^5+18x^4+22x^3+12x^2+24x+12 = \\
=5(x+16)(x+24)(x^5+5x^4+8x^3+27x^2+4x+21)(x^5+17x^4+7x^3+16x^2+23x+13)
\end{gather*}

$|E_1[2](\mathbb{F}_{31})|=1$ since $x^3+3x+1$ is irreducible in $\mathbb{F}_{31}[x]$. $|E_1[3](\mathbb{F}_{31})|=1+2$ the point at infinity and 2 points $(30,\pm \beta)$ where $\beta \in \mathbb{F}_{31}:\beta^2=30^3+3\cdot30+1$. $|E_1[4](\mathbb{F}_{31})|=1$ is trivial since there are no roots of $\frac{\phi_4}{4y}$ is irreducible $\mathbb{F}_{31}[x]$. $\phi_5$ is reducible in $\mathbb{F}_{31}[x]$ but $15^3+3\cdot30+1$ and $7^3+3\cdot7+1$ are not squares in $\mathbb{F}_{31}$ and therefore there are no points with $7$ or $15$ as their $x$ coordinate. This means that also $E_1[5](\mathbb{F}_{31})=1$.

Now we know that since $E_1[3](\mathbb{F}_{31}) \leq E_1(\mathbb{F}_{31})$ that $3|E_1(\mathbb{F}_{31})$. 

So $|E_1(\mathbb{F}_{31})| \in \{21, 24, 27, 30, 33, 36, 39, 42\}$.

Let's look at $E_2$ division polynomials:
\begin{gather*}
\phi_1= 1\\
\phi_2 = 2y\\
\phi_3 = 3x^4+18x^2+19+22 = 3(x+30)(x^3+x^2+7x+3)\\
\phi_4 = 4y(x^6+15x^4+11x^3+17x^2+12x+27)\\
\phi_5 = 5x^{12}+23x^9+16x^8+24x^7+30x^6+2x^5+18x^4+9x^3+12x^2+7x+12 = \\
=5(x+7)(x+15)(x^5+14x^4+7x^3+15x^2+23x+18)(x^5+26x^4+8x^3+4x^2+4x+10)
\end{gather*}

Using the same technique we get that $|E_2[2](\mathbb{F}_{31})|=1, |E_2[3](\mathbb{F}_{31})|=1, |E_2[4](\mathbb{F}_{31})|=1$ and $|E_2[5](\mathbb{F}_{31})|=5$. Therefore $|E_2(\mathbb{F}_{31})| \in \{25, 30, 35, 40\} $.

At the start we noticed that $|E_1(\mathbb{F}_{31})|+|E_2(\mathbb{F}_{31})|=64$. The only solutions are $(|E_1(\mathbb{F}_{31})|, |E_2(\mathbb{F}_{31})|) = (24, 40)$ or $(|E_1(\mathbb{F}_{31})|, |E_2(\mathbb{F}_{31})|) = (39, 25)$.

If the first case was true then $E_1(\mathbb{F}_{31})$ is isomorphic to one of the following:
\begin{gather*}
E_1(\mathbb{F}_{31}) \cong \mathbb{Z}_4 \times \mathbb{Z}_6\\
E_1(\mathbb{F}_{31}) \cong \mathbb{Z}_3 \times \mathbb{Z}_8\\
E_1(\mathbb{F}_{31}) \cong \mathbb{Z}_3 \times \mathbb{Z}_2 \times \mathbb{Z}_2 \times \mathbb{Z}_2\\
\end{gather*}

In the first 2 cases $E_1(\mathbb{F}_{31})$ would have an element of order 4. We have shown that this is not true. If the last case was true then $E_1(\mathbb{F}_{31})$ would have at least 3 involutions but $|E_1[2](\mathbb{F}_{31})|=1$ therefore it must be that $(|E_1(\mathbb{F}_{31})|, |E_2(\mathbb{F}_{31})|) = (39, 25)$.

\end{document}