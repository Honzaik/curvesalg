\documentclass[12pt, a4paper]{article}
\usepackage[margin=1in]{geometry}
\usepackage[utf8x]{inputenc}
\usepackage{indentfirst} %indentace prvního odstavce
\usepackage{mathtools}
\usepackage{amsfonts}
\usepackage{amsmath}
\usepackage{amssymb}
\usepackage{graphicx}
\usepackage{enumitem}
\usepackage{subfig}
\usepackage{float}
\usepackage[czech]{babel}
\usepackage{mathdots}

\begin{document}
\begin{center}
\large NMMB430 - DÚ 3

\normalsize Jan Oupický
\end{center}
\vspace{1\baselineskip}

\section{}
We will proceed by calculating $[2]P, [2]P+[2]P = [4]P$ (doubling) and finally $[4]P+P=[5]P$ (addition).

$[2]P$:
\begin{gather*}
\gamma_1 = 20, \gamma_2 = 5, \gamma_3 = 8\\
\implies
[2]P = (20:5:8)
\end{gather*}

$[4]P$:
\begin{gather*}
\gamma_1 = 18, \gamma_2 = 4, \gamma_3 = 4\\
\implies
[4]P = (18:4:4)
\end{gather*}

$[4]P+P$:
\begin{gather*}
U = 0, W = 0, V = 26\\
\implies
[5]P = (0:-4:4)
\end{gather*}

Since $4^{-1} = 8$ we get that $[5]P = (0:-4:4) = (0:-1:1)$.
\end{document} 