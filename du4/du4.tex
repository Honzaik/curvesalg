\documentclass[12pt, a4paper]{article}
\usepackage[margin=1in]{geometry}
\usepackage[utf8x]{inputenc}
\usepackage{indentfirst} %indentace prvního odstavce
\usepackage{mathtools}
\usepackage{amsfonts}
\usepackage{amsmath}
\usepackage{amssymb}
\usepackage{graphicx}
\usepackage{enumitem}
\usepackage{subfig}
\usepackage{float}
\usepackage[czech]{babel}
\usepackage{mathdots}

\begin{document}
\begin{center}
\large NMMB430 - DÚ 4

\normalsize Jan Oupický
\end{center}
\vspace{1\baselineskip}

\section{}
Let $M$ be a Montgomery curve given by $f(x,y) = By^2 - (x^3+Ax^2+x)$ where $A,B \in K, B \neq 0$. The partial derivatives of $f$ are:
\begin{gather*}
\frac{\partial f}{\partial x} (x,y) = -3x^2-2Ax-1\\
\frac{\partial f}{\partial y} (x,y) = 2By\\
\end{gather*}

Since $B \neq 0$ we see that a singular point $(\alpha_1, \alpha_2) \in M$ has to satisfy $\alpha_2 = 0$. Therefore candidates for a singular point on $M$ are only points $(\alpha_1, 0) \in M$. There are at most $3$ such points of $M$ since $\alpha_1$ has to satisfy $f(\alpha_1, 0) = \alpha_1^3+A\alpha_1^2+\alpha_1 = \alpha_1(\alpha_1^2+A\alpha_1 + 1) = 0$. Clearly $(0,0)$ is a point of $M$ but it is not singular since $\frac{\partial f}{\partial x} (0,0)\neq 0$.

The other 2 points are $(\xi_1, 0), (\xi_2, 0)$ where $\xi_{1,2}$ are the roots of $x^2+Ax+1$ i.e.
\begin{gather*}
\xi_1 = \frac{-A+\sqrt{A^2-4}}{2}, \xi_2 = \frac{-A-\sqrt{A^2-4}}{2}
\end{gather*}

For these points to be singular they also have satisfy $\frac{\partial f}{\partial x}(\xi_{1,2}, 0) = 0$ i.e. $\xi_{1,2}$ have to also be roots $\xi_{1,2}'$ of $-3x^2-2Ax-1$ which are:
\begin{gather*}
\xi'_1 = \frac{-A+\sqrt{A^2-3}}{3}, \xi'_2 = \frac{-A-\sqrt{A^2-3}}{3}
\end{gather*}

Now we want to know what $A$ has to be for $\xi_1 = \xi'_1$ or $\xi_1 = \xi'_2$ and same for $\xi_2$.

$\xi_1 = \xi'_1$:
\begin{gather*}
\frac{-A+\sqrt{A^2-4}}{2} = \frac{-A+\sqrt{A^2-3}}{3} \iff -A+3\sqrt{A^2-4} = 2\sqrt{A^2-3} \implies\\
A^2-4 = A\sqrt{A^2-4} \implies A^2=4 \iff A = \pm 2
\end{gather*}
We have shown that if $\xi_1 = \xi'_1$ then $A = 2$ or $A = -2$. In this case only $A = -2$ works. For the case $\xi_1 = \xi'_2$ $A$ has to be equal to $2$. Cases $\xi_2 = \xi'_{1,2}$ give the same conditions. Since we are considering only $A= \pm2$ we see that in these cases $\xi_1 = \xi_2$.

We have shown that $M$ is singular iff $A = \pm 2$. In both cases the affine singular point is $(\frac{-A}{2}, 0)$.

The projective curve $\hat{M}$ is given by $F(X,Y,Z) = BY^2Z-X^3-AX^2Z-XZ^2$. The only projective point with $Z=0$ is $(0:1:0)$. Using the partial derivatives:
\begin{gather*}
\frac{\partial F}{\partial X}(X,Y,Z) = -3X^2-2AXZ-Z^2\\
\frac{\partial F}{\partial Y}(X,Y,Z) = 2BYZ\\
\frac{\partial F}{\partial Z}(X,Y,Z) = BY^2-AX^2-2XZ\\
\end{gather*}
Since $\frac{\partial F}{\partial Z}(0,1,0) = B \neq 0$ we see that $\hat{M}$ is smooth at $(0:1:0)$ i.e. $M$ is smooth at the point at infinity.

We have shown that if $M$ is singular the only singularity is at $(\frac{-A}{2}, 0)$.

\section{}
Since $\alpha = (\alpha_1, \alpha_2) \neq (0,0)$ we can assume that $\alpha_1 \neq 0$ because if $\alpha_1 = 0$ and $\alpha \in M$ then $\alpha_2=0$. Using the formula $\alpha \tilde{\oplus} \beta, \alpha \neq \beta$ for Montgomery curve in a case when $\beta = (0,0)$ we have:
\begin{gather*}
(\gamma_1, \gamma_2) = \alpha \tilde{\oplus} (0,0)\\
\tilde{\lambda} = \frac{-\alpha_2}{-\alpha_1} = \frac{\alpha_2}{\alpha_1} \implies\\
\gamma_1 = -\alpha_1 +B\left(\frac{\alpha_2}{\alpha_1}\right)^2-A = \frac{-\alpha_1^3+B\alpha_2^2-A\alpha_1^2}{\alpha_1^2}\\
\text{Substituting $B\alpha_2^2 = \alpha_1^3+A\alpha_1^2+\alpha_1$:}\\
\gamma_1 = \frac{\alpha_1}{\alpha_1^2} = \frac{1}{\alpha_1}\\
\gamma_2 = \frac{\alpha_2}{\alpha_1}\left(\alpha_1 - \frac{1}{\alpha_1}\right)-\alpha_2 = \frac{-\alpha_2}{\alpha_1^2} \implies\\
(\alpha_1, \alpha_2) \tilde{\oplus} (0,0) = \left(\frac{1}{\alpha_1}, \frac{\alpha_2}{\alpha_1^2}\right)
\end{gather*}

\section{}
Using $M.6$ we can find curves which are surely $\mathbb{Z}_5$-equivalent to a Montgomery curve. Those are:
\begin{gather*}
y^2=x^3+x\\
y^2=x^3+4x
\end{gather*}
These curves are also Montgomery curves with $A=0, B = 1$. 

For the rest we will use $M.5$. Here is a list of the polynomials and their roots in $\mathbb{Z}_5$ and if the roots satisfy the condition in $M.5$ we write the corresponding Montgomery curve:
\begin{align*}
& x^3+1,\{4\}, f'(4) = 3 \notin (\mathbb{Z}_5^*)^2\\
& x^3+2, \emptyset \\
& x^3 + x + 1, \emptyset \\
& x^3 + x + 2, \{4\}, f'(4) = 4 \in (\mathbb{Z}_5^*)^2 \approx 2y^2=x^3+x^2+x\\
& x^3 + 2x, \{0\}, f'(0) = 2 \notin (\mathbb{Z}_5^*)^2\\
& x^3 + 2x+1, \emptyset\\
& x^3 +3x, \{0\}, f'(0) = 3 \notin (\mathbb{Z}_5^*)^2\\
& x^3 +3x+2, \emptyset\\
& x^3 +4x+1, \{3\}, f'(3)=1 \in (\mathbb{Z}_5^*)^2 \approx y^2=x^3+4x^2+x\\
& x^3 +4x+2, \emptyset
\end{align*}

In total we have 4 curves which are $\mathbb{Z}_5$-equivalent to Montgomery curves $y^2=x^3+x, y^2=x^3+4x, y^2=x^3 + x + 2, y^2=x^3 +4x+1$.

\section{}
$102_{10} = 1100110_2 \implies$
\begin{gather*}
n_1 = 1_2 = 1\\
n_2 = 11_2 = 3\\
n_3 = 110_2 = 6\\
n_4 = 1100_2 = 12\\
n_5 = 11001_2 = 25\\
n_6 = 110011_2 = 51\\
n_7 = 102
\end{gather*}

\end{document}