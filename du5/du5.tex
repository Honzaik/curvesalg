\documentclass[12pt, a4paper]{article}
\usepackage[margin=1in]{geometry}
\usepackage[utf8x]{inputenc}
\usepackage{indentfirst} %indentace prvního odstavce
\usepackage{mathtools}
\usepackage{amsfonts}
\usepackage{amsmath}
\usepackage{amssymb}
\usepackage{graphicx}
\usepackage{enumitem}
\usepackage{subfig}
\usepackage{float}
\usepackage[czech]{babel}
\usepackage{mathdots}

\begin{document}
\begin{center}
\large NMMB430 - DÚ 5

\normalsize Jan Oupický
\end{center}
\vspace{1\baselineskip}

\section{}
Using E.7 we get that we have to set $a = \frac{A-2}{B}, d = \frac{A+2}{B}$ which we can always do since $B \neq 0$ and $A\neq \pm2$ and therefore $a \neq d$ and $a, d \in K^*$. Here are the Montgomery curves from the previous exercise and their birationally equivalent Edwards counterparts. We are working with $K = \mathbb{Z}_5$:
\begin{gather*}
M: 2y^2=x^3+x^2+x \implies (A,B) = (1,2) \implies\\
(a,d) = \left(-\frac{1}{2}, \frac{3}{2}\right) = (2,4) \implies E: 2x^2+y^2=1+4x^2y^2 \\\\
M: y^2=x^3+4x^2+x \implies (A,B) = (4, 1) \implies\\
(a,d) = (2, 1) \implies E: 2x^2+y^2=1+x^2y^2\\\\
M: y^2=x^3+x \implies (A,B) = (0, 1) \implies\\
(a, d) = (3, 2) \implies E: 3x^2+y^2=1+2x^2y^2\\
\text{since $3 = 2^3 = 2\cdot2^2$ in $\mathbb{Z}_5$ we get $E$ is $\mathbb{Z}_5$-equivalent to $E'$:}\\
E': 2x^2+y^2=1+3x^2y^2\\\\
M: 2y^2=x^3+x \implies (A,B) = (0,2) \implies\\
(a, d) = (4,1) \implies E: 4x^2+y^2=1+x^2y^2\\
\text{which is $\mathbb{Z}_5$-equivalent to $E'$:}\\
E': x^2+y^2=1+4x^2y^2
\end{gather*}

\section{}
First assume $d > 0$. Since $a < 0$ we can easily see that $y^2 = 1+dx^2y^2-ax^2 \geq 1 \implies$ all points $(\alpha, \beta)$ on the curve must have $|\beta| \geq 1$. Also we can express the equation as $y^2=\frac{1-ax^2}{1-dx^2} \implies y = \pm \sqrt{\frac{1-ax^2}{1-dx^2}}$. Real points therefore must satisfy $\frac{1-a\alpha^2}{1-d\alpha^2} \geq 0$. The numerator is always positive. Therefore we have to only look at the denominator. We have condition $1-d\alpha^2 > 0$ and this happens only if $|\alpha| < \sqrt{\frac{1}{d}}$. From this we can conclude that the curve is bounded by lines $x=\pm \sqrt{\frac{1}{d}}$. It is symmetrical with respect to the lines $x=0$ and $y=0$. It has therefore 2 parts. If we look at the part where $\beta \geq 1$ it has a U-shape form bounded by the lines mentioned. The part $\beta \leq -1$ looks similar since it is symmetrical. Changing the $a$ parameter affects the "narrowness" of the U-shape. Changing the parameter $d$ changes the bounds of the U-shape.

Now assuming $d < 0$. We still have $y = \pm \sqrt{\frac{1-ax^2}{1-dx^2}}$ which is defined everywhere since both the denominator and numerator is positive. It is symmetrical with respect to the lines $x=0$ and $y=0$. We have to differentiate between cases when $a < d$ and $a > d$.

First assuming $a < d$ we get that $\frac{1-ax^2}{1-dx^2} \geq 1 \implies |\beta| \geq 1$. But we also have an upper bound since $\lim_{x \to \infty} \sqrt{\frac{1-ax^2}{1-dx^2}} = \sqrt{\frac{a}{d}}$. Considering only the part $\beta \geq 1$ the curve has a V-shape upper bounded by the line $y=\sqrt{\frac{a}{d}}$. Similarly the part $\beta \leq -1$ is lower bounded by the line $y=-\sqrt{\frac{a}{d}}$

The case $a > d$ means that $\frac{1-ax^2}{1-dx^2} \leq 1 \implies |\beta| \leq 1$ and the limit is the same but $\sqrt{\frac{a}{d}} < 1$ therefore it is a lower bound. Considering the case for $\beta > \sqrt{\frac{a}{d}}$ we get that the shape is a flipped $V$ with an upper bound $y=1$ and the mentioned lower bound $y=\sqrt{\frac{a}{d}}$ and symetrically for the case $\beta < -\sqrt{\frac{a}{d}}$.

\end{document}