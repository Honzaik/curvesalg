\documentclass[12pt, a4paper]{article}
\usepackage[margin=1in]{geometry}
\usepackage[utf8x]{inputenc}
\usepackage{indentfirst} %indentace prvního odstavce
\usepackage{mathtools}
\usepackage{amsfonts}
\usepackage{amsmath}
\usepackage{amssymb}
\usepackage{graphicx}
\usepackage{enumitem}
\usepackage{subfig}
\usepackage{float}
\usepackage[czech]{babel}
\usepackage{mathdots}
\usepackage{slashbox}

\begin{document}
\begin{center}
\large NMMB430 - DÚ 1

\normalsize Jan Oupický
\end{center}
\vspace{1\baselineskip}

\section{}
I don't see any issues with the proof or the lemma statement.

\section{}
The set mentioned corresponds to a curve $C= V_f$ where $f=y^2-x^3+1$. From ex. 1) we know this polynomial is irreducible in $K$. By definition $C$ is an irreducible planar curve.
\begin{gather*}
\frac{\partial f}{\partial x} (x,y) = -3x^2\\
\frac{\partial f}{\partial x} (x,y) = 2y
\end{gather*}

First we consider char$(K) \notin \{2,3\}$. To achieve singularity $y$ has to be $0$. The only points on the curve that have $y$ coordinate equal to $0$ are the 3rd roots of unity in $K$. We also need to have $-3x^2 = 0$ but this is impossible since the 3rd roots of unity are not $0$ in $K$ by definition.

Now assuming char$(K) \in \{2,3\}$ we see that for each case we get one partial derivative to be 0 at all points. From that we can conclude that we only need to find a point with $x=0$ (for char$(K)=2$) or with $y=0$ for char$(K)=3$. We can calculate the corresponding point on the curve from the polynomial which defines the curve.

Therefore the affine curve is singular iff char$(K) \in \{2,3\}$.

The projective curve is given by $F= Y^2Z-X^3+Z^3$. There exist only one point in $\hat{C} = V_F$ s.t. $Z = 0$ (the point at infinity) and it has coordinates $\mathcal{O} = (0:1:0)$ since if we set $Z=0$ then $X$ has to be $0$ as well and $Y$ has to be nonzero (for it to be a point in $\mathbb{P}^2$).
\begin{gather*}
\frac{\partial F}{\partial X} (X,Y,Z) = -3X^2\\
\frac{\partial F}{\partial Y} (X,Y,Z) = 2YZ\\
\frac{\partial F}{\partial Z} (X,Y,Z) = Y^2+3Z^2
\end{gather*}
We can see that the partial derivative w.r.t. $Z$ is not $0$ if we plug in $\mathcal{O}$. This shows that if char$(K)$ is not 2 or 3 the projective curve is also smooth since we have shown it is smooth at all affine points studying the affine curve and shown smoothness at the only projective point which does not correspond to an affine point.

\section{}
Assume char$(K) \neq 3$. We have $\sigma \in K(C)$ where $C = V_f, f = y^2-x^3+1$. $\sigma = \frac{x-1+(f)}{y+(f)}$. By definition of $K(C)$ this equality holds $y^2+(f)=x^3-1+(f)$ in $K(C)$.
\begin{gather*}
\frac{x-1+(f)}{y+(f)} = \frac{y(x-1)+(f)}{y^2+(f)} = \frac{y(x-1)+(f)}{x^3-1+(f)} = \\
=\frac{y(x-1)+(f)}{(x-1)(x^2+x+1)+(f)} = \frac{y + (f)}{x^2+x+1+(f)}
\end{gather*}
We have expressed $\sigma$ using a different representative $\frac{y}{x^2+x+1}$. Clearly the denominator is not $0$ at point $(1,0)$ which means $\sigma$ is defined at $(1,0)$ and $\sigma(1,0)=0$.

\section{}
From ex. 1) we know that the polynomial $f(x,y) = y^2-x^3+1 \in K[x,y]$ is irreducible over $K$. We also know that this happens if and only if $F(X,Y,Z) = Y^2Z-X^3+Z^3 \in K[X,Y,Z]$ is irreducible over $K$. 

Denote $g(x,y) = x^3+x+y^3 \in K[x,y]$ and it's corresponding homogenized polynomial as $G(X,Y,Z) = X^3+XZ^2-Y^3$. Denote by $\psi$ an automorphism of the ring $K[X,Y,Z]$ s.t. $\psi(a(X,Y,Z)) \mapsto a(Y,Z,X)$ ($\psi$ just permutes the variables). $\psi$ is clearly an automorphism from the definition since it only permutes the variables and therefore it must map irreducible polynomials upon irreducible polynomials. Now we observe that $\psi(F) = G$. $F$ is irreducible and therefore also $G$ is irreducible which is irreducible iff $g$ is irreducible.
\begin{gather*}
\frac{\partial g}{\partial x} = 3x^2+1\\
\frac{\partial g}{\partial y} = 3y^2
\end{gather*}

Set $D=V_g$. To have a singularity we need $y=0$. The points of $D$ with $y=0$ are such that for their $x$ coordinate it must be that $x^3+x=0 \iff x(x^2+1)=0$. This gives us at most 3 potential points. The number of points depends on the existence of $a \in K$ s.t. $a^2 = -1$. 

Point $(0,0)$ which exists always is not a singular point since the partial derivative w.r.t. $x$ is not $0$.

Now assuming there exists $a \in K$ s.t. $a^2=-1$ we have $2$ other points of $P_{1,2}=(\pm a, 0) \in D$. The partial derivative w.r.t. $x$ is $3a^2+1 = -3+1 = -2$. If char$(K)\neq 2$ we see that $D$ is smooth.

A projective version of $D$ ($\hat{D} = V_G$) has also a point at infinity $(1:1:0)$ which is smooth assuming char$(K)\neq 3$.

To sum it up: $D$ is a smooth affine curve $\iff$ char$(K) \neq 2$ and $\hat{D}$ is a smooth projective curve $\iff$ char$(K)\notin \{2,3\}$.
\end{document} 