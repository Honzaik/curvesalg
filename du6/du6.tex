\documentclass[12pt, a4paper]{article}
\usepackage[margin=1in]{geometry}
\usepackage[utf8x]{inputenc}
\usepackage{indentfirst} %indentace prvního odstavce
\usepackage{mathtools}
\usepackage{amsfonts}
\usepackage{amsmath}
\usepackage{amssymb}
\usepackage{graphicx}
\usepackage{enumitem}
\usepackage{subfig}
\usepackage{float}
\usepackage[czech]{babel}
\usepackage{mathdots}

\begin{document}
\begin{center}
\large NMMB430 - DÚ 6

\normalsize Jan Oupický
\end{center}
\vspace{1\baselineskip}

\section{}
Twisted Edwards curve in completed coordinates $\mathbb{P}^1 \times \mathbb{P}^1$ is given by the equation $aX_1^2Y_2^2+Y_1^2X_2^2 = X_2^2Y_2^2 + dX_1^2Y_1^2$ ($X$s correspond to the first part of the point and $Y$s correspond to the second part). Since affine points $(\alpha, \beta)$ of a Twisted Edwards curve are 1-1 mapped upon points $((\alpha: 1), (\beta : 1))$ the points at infinity must have $X_2 = 0$ or $Y_2 = 0$.

If $X_2 = 0$ then $X_1 \neq 0$ since $(0:0) \notin \mathbb{P}^1$. Then the equation is $aX_1^2Y_2^2 = dX_1^2Y_1^2$. Since $X_1 \neq 0$ then it can be simplified into $aY_2^2 = dY_1^2$. $a,d \in K^*$ then $\frac{a}{d} = \left(\frac{Y_1}{Y_2}\right)^2$ $Y_2$ is also not 0 since that would imply $Y_1=0$ as well. Let $\frac{a}{d}=t^2$ then we have points $((1:0),(\pm t:1))$.

Other case $Y_2 = 0$ implies $Y_1 \neq 0$. The equation is $X_2^2 = dX_1^2$. Again $X_1,X_2 \neq 0$ which implies $d = s^2$ and the points are $((1:\pm s),(1:0))$. We have exhausted all possibilities.

Therefore as said in the lecture. The group has 0,2 or 4 points at infinity depending on $ad^{-1}, d$ being squares in $K$. If both are squares then we have 4 points. If none are then 0 points. If one of them is a square then we have 2.

\section{}
Consider the curve $E: 2x^2+y^2=1+3x^2y^2$ over $\mathbb{Z}_5$. The only affine points on the curve are $(0,1), (0,4)$. Since $ad^{-1}=4=2^2$ but $d$ is not a square, we have 2 points at infinity. $(0,1)$ has order 1, $(0,4) \oplus (0,4) = (0,1)$ i.e. $(0,4)$ has order $2$. By using the addition formula in complete coordinates we get that the 2 points at infinity are of order 2 as well. Therefore $|E(\mathbb{Z}_5)| = 4$ and has 3 points of order 2 and 1 point of order 1. This is an example of a twisted edwards curve with no point of order 4 over $K$.

\section{}
We will analyze how many points at infinity our curves have. We have already analyzed one of then in 2). Remaining ones are:

$E: 2x^2+y^2=1+4x^2y^2$: $d=4$ is a square. $ad^{-1} = 2\cdot 4 = 3$ is not a square. We have therefore 2 points at infinity in the group $E(\mathbb{Z}_5)$. In completed coordinates they are $((1:2), (1:0)), ((1:3),(1:0))$.

$E: 2x^2+y^2=1+4x^2y^2$: $d=1$ is a square. $ad^{-1} = 2\cdot 1 = 2$ is not a square. We have therefore 2 points at infinity in the group $E(\mathbb{Z}_5)$. In completed coordinates they are $((1:1), (1:0)), ((1:4),(1:0))$.

$E:x^2+y^2=1+4x^2y^2$: $d=4$ is a square. $ad^{-1} = 1\cdot 4 = 4$ is a square. We have therefore 4 points at infinity in the group $E(\mathbb{Z}_5)$. In completed coordinates they are $((1:2), (1:0)), ((1:3),(1:0)), ((1:0), (2:1)), ((1:0), (3:1))$.
\end{document}